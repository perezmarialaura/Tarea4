\documentclass[a4paper,10pt]{article}
\usepackage[T1]{fontenc}
\usepackage[utf8]{inputenc}
\usepackage{float}
\usepackage{tikz}

\begin{document}

\title{Resultados Tarea 4}

\author{María Laura Pérez Lara}

\maketitle
\section{Solución de ecuación de onda 1D: cuerda con extremos fijos}

Al solucionar la ecuación de onda con extremos fijos de una guitarra con velocidad de propagación $c=250 m/s$ y longitud $L=0.64m$, se obtienen las siguientes gráficas de $\phi$ para diferentes tiempos, donde la función tiene un periodo aproximado de 5.13 ms:

\begin{figure}[H]
    \centering
    \includegraphics[width=0.7\textwidth]{plots_guitarrafija.png}
    \caption{Función de onda de una cuerda con extremos fijos para diferentes tiempos.}
\end{figure}

\section{Solución de ecuación de onda 1D: cuerda con un extremo fijo}

Ahora se tiene una condición inicial de amplitud 0 en toda la cuerda, y se aplica una perturbación dependiente del tiempo en un extremo de la forma $\sin{2\pi ct/L}$ donde $c$ es la velocidad de propagación y $L$ la longitud de la cuerda (ambas igual que antes). El periodo en este caso es de 0.00256 s. La gráfica que muestra la función de onda para diferentes tiempos es la siguiente:

\begin{figure}[H]
    \centering
    \includegraphics[width=0.7\textwidth]{plots_cuerdasuelta.png}
    \caption{Función de onda de una cuerda con un extremo para diferentes tiempos.}
\end{figure}

\section{Solución de ecuación de onda 2D: tambor con extremos fijos}

Se tiene la membrana de un tambor cuadrado de lado 0.5 m con condición inicial gaussiana. Las gráficas que muestran la función de onda para diferentes tiempos son las siguientes:

\begin{figure}[H]
    \centering
    \includegraphics[width=0.7\textwidth]{Tambor_init.png}
    \caption{Condición inicial del tambor.}
\end{figure}

\begin{figure}[H]
    \centering
    \includegraphics[width=0.7\textwidth]{Tambor_toctavos.png}
    \caption{Membrana después de T/8.}
\end{figure}

\begin{figure}[H]
    \centering
    \includegraphics[width=0.7\textwidth]{Tambor_tcuartos.png}
    \caption{Membrana después de T/4.}
\end{figure}

\begin{figure}[H]
    \centering
    \includegraphics[width=0.7\textwidth]{Tambor_tmedios.png}
    \caption{Membrana después de T/2.}
\end{figure}

\end{document}
